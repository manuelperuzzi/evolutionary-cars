% !TeX spellcheck = it_IT
\documentclass[a4paper,12pt]{article}

\usepackage{alltt, fancyvrb, url}
\usepackage{graphicx}
\usepackage{algorithmic}
\usepackage[utf8]{inputenc}
\usepackage{titling}
\usepackage{fancyhdr}
\usepackage{fontenc}
\usepackage{amsmath,mathtools,algorithm}
\usepackage{amssymb}
\usepackage{longtable}
\usepackage{setspace}
\usepackage{listings}
\usepackage{color}
\usepackage{eurosym}
\usepackage{array}
\usepackage[referable]{threeparttablex}
\usepackage{pifont}

\newcommand{\cmark}{\ding{51}}
\newcommand{\xmark}{\ding{55}}

\usepackage[italian,hidelinks]{hyperref}

\usepackage[italian]{babel}
\usepackage[italian]{cleveref}


\pretitle{%
	\begin{center}
		\LARGE
	}
\posttitle{\end{center}}


\title{\Huge \textbf{Evolutionary Cars} \\
	\vspace{10pt}
	\vspace{20pt}
}
\author{
	Gabriele Graffieti \\ \small \url{gabriele.graffieti@studio.unibo.it}
	\vspace{15pt}
	\\
	Alfredo Maffi \\ \small \url{alfredo.maffi@studio.unibo.it}
	\vspace{15pt}
	\\
	Manuel Peruzzi \\ \small \url{manuel.peruzzi@studio.unibo.it}
}

\date{}

\begin{document}

\maketitle
\pagenumbering{arabic}
\newpage
\tableofcontents
\newpage

\section{Introduzione}

Questo documento è la relazione del progetto Evolutionary Cars, realizzato per il corso di Sistemi Intelligenti Robotici, erogato dalla facoltà di Ingegneria e Scienze Informatiche dell'Università di Bologna (A.A. 2017/2018).

\section{Stato dell'arte}

\section{Architettura}

\section{Evoluzione}

\section{Simulazione}

\section{Risultati}

\section{Conclusioni}

\end{document}
